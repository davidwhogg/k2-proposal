% This file is part of the OpenK2 project.
% Copyright 2015 the authors.

% ## style notes:
% - Use \textbf{} for emphasis of key deliverables or proposed items.

% ## issues:
% - [Put issues here or on github].

\documentclass[12pt]{article}
\setlength{\headheight}{2ex}
\setlength{\headsep}{3ex}
\input{hogg_nasa}
\pagestyle{myheadings}
\markright{\textsf{\shortauthor~/~\fulltitle}}

\begin{document}

Hello World!  HOGG:  Introduction text here.

What is \ketu\ doing and why?... (Don't emphasize the reaction-wheel failure.)

What is different about \ketu\ from \kepler\ Main Mission?

What are we going to learn from \ketu\ that we didn't learn from the Main Mission?

How does all this fit into the long-term goals of NASA Origins?

\section{The first Open \ketu\ Catalog}

Hello World!  DFM:  Summary of our paper here, with figures.

\section{Development and enhancements}

Hello World!  We need a better title for this section

DFM:  Laundry list here (to start)

\section{Population inferences}

Hello World!

\section{Prior NASA support}

Do we need this section?

\section{Culture change in exoplanet science}

This section title is too strident.

Everything in the group is done out in the open.
Refer to this proposal's github repo and tag.

\section{Management plan and data release}

The project is extremely streamlined:

The postdoctoral scholar will do most of the heavy lifting.

Foreman-Mackey will supervise on all methodological and exoplanetary
matters.

Hogg will be responsible for ensuring that catalogs get released and
papers get written.

Enhancements will proceed in parallel with Catalog generation;
Catalogs will be generated for each Campaign as it is released.
Then, each year, we will build updated catalogs for all prior
Campaigns, incorporating all enhancements to date.

Everything will be strictly versioned, so that there will be no
ambiguity about which Catalog is being used where.

\end{document}
