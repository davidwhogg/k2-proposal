% This file is part of the OpenK2 project.
% Copyright 2015 the authors.

% ## style notes:
% - Use \textbf{} for emphasis of key deliverables or proposed items.

% ## issues:
% - [Put issues here or on github].

\documentclass[12pt,preprint]{aastex}
\setlength{\headheight}{2ex}
\setlength{\headsep}{3ex}
%----- exact 1-in margins
% NB: headheight and headsep MUST exist and be set
\setlength{\textwidth}{6.5in}
\setlength{\textheight}{9in}
\addtolength{\textheight}{-1.0\headheight}
\addtolength{\textheight}{-1.0\headsep}
\setlength{\topmargin}{0.0in}
\setlength{\oddsidemargin}{0.0in}
\setlength{\evensidemargin}{0.0in}

%----- typeset certain kinds of words
\newcommand{\observatory}[1]{\textsl{#1}}
\newcommand{\package}[1]{\textsf{#1}}
\newcommand{\project}[1]{\textsl{#1}}
\newcommand{\an}{\package{Astrometry.net}}
\newcommand{\NASA}{\observatory{NASA}}
\newcommand{\Kepler}{\observatory{Kepler}}
\newcommand{\kepler}{\Kepler}
\newcommand{\ketu}{\observatory{K2}}
\newcommand{\MAST}{\observatory{MAST}}
\newcommand{\EA}{\observatory{Exoplanet Archive}}
\newcommand{\TESS}{\observatory{TESS}}
\newcommand{\galex}{\observatory{GALEX}}
\newcommand{\Spitzer}{\observatory{Spitzer}}
\newcommand{\gaia}{\observatory{Gaia}}
\newcommand{\Gaia}{\gaia}
\newcommand{\lsst}{\observatory{LSST}}
\newcommand{\sdss}{\observatory{SDSS}}
\newcommand{\latin}[1]{\textit{#1}}
\newcommand{\eg}{\latin{e.g.}}
\newcommand{\etal}{\latin{et~al.}}
\newcommand{\etc}{\latin{etc.}}
\newcommand{\ie}{\latin{i.e.}}
\newcommand{\vs}{\latin{vs.}}

%----- typeset journals
% \newcommand{\aj}{Astron.\,J.}
% \newcommand{\apj}{Astrophys.\,J.}
% \newcommand{\apjl}{Astrophys.\,J.\,Lett.}
% \newcommand{\apjs}{Astrophys.\,J.\,Supp.\,Ser.}
% \newcommand{\mnras}{Mon.\,Not.\,Roy.\,Ast.\,Soc.}
% \newcommand{\aap}{Astron.\,\&~Astrophys.}

%----- Tighten up paragraphs and lists
\setlength{\parskip}{0.0ex}
\setlength{\parindent}{0.2in}
\renewenvironment{itemize}{\begin{list}{$\bullet$}{%
  \setlength{\topsep}{0.0ex}%
  \setlength{\parsep}{0.0ex}%
  \setlength{\partopsep}{0.0ex}%
  \setlength{\itemsep}{0.0ex}%
  \setlength{\leftmargin}{1.5\parindent}}}{\end{list}}
\newcounter{actr}
\renewenvironment{enumerate}{\begin{list}{\arabic{actr}.}{%
  \usecounter{actr}%
  \setlength{\topsep}{0.0ex}%
  \setlength{\parsep}{0.0ex}%
  \setlength{\partopsep}{0.0ex}%
  \setlength{\itemsep}{0.0ex}%
  \setlength{\leftmargin}{1.5\parindent}}}{\end{list}}

%----- mess with paragraph spacing!
\makeatletter
\renewcommand\paragraph{\@startsection{paragraph}{4}{\z@}%
                                    {1ex}%
                                    {-1em}%
                                    {\normalfont\normalsize\bfseries}}
\makeatother

%----- Special Hogg list for references
  \newcommand{\hogglist}{%
    \rightmargin=0in
    \leftmargin=0.25in
    \topsep=0ex
    \partopsep=0pt
    \itemsep=0ex
    \parsep=0pt
    \itemindent=-1.0\leftmargin
    \listparindent=\leftmargin
    \settowidth{\labelsep}{~}
    \usecounter{enumi}
  }

%----- side-to-side figure macro
%------- make numbers add up to 94%
 \newlength{\figurewidth}
 \newlength{\captionwidth}
 \newcommand{\ssfigure}[3]{%
   \setlength{\figurewidth}{#2\textwidth}
   \setlength{\captionwidth}{\textwidth}
   \addtolength{\captionwidth}{-\figurewidth}
   \addtolength{\captionwidth}{-0.02\figurewidth}
   \begin{figure}[htb]%
   \begin{tabular}{cc}%
     \begin{minipage}[c]{\figurewidth}%
       \resizebox{\figurewidth}{!}{\includegraphics{#1}}%
     \end{minipage} &%
     \begin{minipage}[c]{\captionwidth}%
       \textsf{\caption[]{\footnotesize {#3}}}%
     \end{minipage}%
   \end{tabular}%
   \end{figure}}

%----- top-bottom figure macro
 \newlength{\figureheight}
 \setlength{\figureheight}{0.75\textheight}
 \newcommand{\tbfigure}[2]{%
   \begin{figure}[htp]%
   \resizebox{\textwidth}{!}{\includegraphics{#1}}%
   \textsf{\caption[]{\footnotesize {#2}}}%
   \end{figure}}

%----- deal with pdf page-size stupidity
\special{papersize=8.5in,11in}
\setlength{\pdfpageheight}{\paperheight}
\setlength{\pdfpagewidth}{\paperwidth}

% specials for this proposal
\newcommand{\shortauthor}{Hogg \& Foreman-Mackey}
\newcommand{\fulltitle}{The Open \ketu\ Catalog and Exoplanet Populations}
\newcommand{\catalogname}{\observatory{The Ketu Catalog}}

% no more bad lines!
\sloppy\sloppypar

\pagestyle{myheadings}
\markright{\textsf{\shortauthor~/~\fulltitle}}
\usepackage{url}
\usepackage{hyperref}

\begin{document}

\emph{The \kepler\ Spacecraft changed the world.}

Or the world of exoplanets, at least.
It found thousands of exoplanet candidates and more than a thousand
confirmed exoplanets.
With a four-year mission and part-in-$10^5$ precision time-domain
measurements of the brightnesses of $10^5$ stars, it is sensitive to
transits of true Earth analog planets (planets on year-ish orbits with
Earth-ish radii around Sun-like stars).
It hasn't quite found a perfect Earth analog, but it has found planets
in a wide range of orbits and with a wide range of properties, and
also done a huge amount to constrain the abundance of Earth-like and
(potentially) habitable planets.

Now that the \kepler\ Main Mission is over (ended by the failure of a
second reaction wheel), the Spacecraft is operating in a new mode,
\kt.
In this mode, the Spacecraft observes a set of 12 fields around the
ecliptic plane, each for about 80 days at a time.
\kt\ is observing for less time on any particular star than in the
Main Mission, but it is covering a wider range of stars in a wider
range of environments.
For example, the \kt\ Campaigns include cluster stars as well as field
stars, which opens up significant discovery space (and significantly
complexifies data-analysis pipelines).
For another example, it is focusing on lower-mass stars, which are now
known to have more abundant planets than Sun-like stars and which
serve as hosts to planets on shorter-period orbits.
In the course of its 12 Campaigns---its observations of its 12
fields---\kt\ data will contain an abundance of exoplanets as
impressive and important as that from the original \kepler\ Main
Mission.

There is one missing piece, however:
From each Campaign, the \kt\ project is delivering image patches (one
patch per star or star cluster every 30~min for 80 to 90~days), but it
is not delivering lists of detected exoplanets.
Unlike the \kepler\ Main Mission, the \kt\ deliverables are only raw
data.
Here \textbf{we propose to deliver the exoplanet catalog} for \kt,
along with the information and tools required to use it responsibly
for statistical studies of exoplanets.

There are also interesting and valuable technical challenges brought
by the \kt\ data.
One is that the Spacecraft pointing is much less stable for \kt\ than
it was for the \kepler\ Main Mission.
This means that far more attention must be paid to instrument-induced
effects, especially when we care (as we do) about very small planets.
This accounting for instrumental noise (or calibration, if you will)
is our specialty, and the thing we do best in the community.

Another interesting technical challenge relates to the brief duration
of the \kt\ Campaigns.
Coming in at a few months, they are intermediate between \kepler\ Main
Mission (four years) and the upcoming NASA \tess\ Mission (one month for most of the sky).
This means that even short-period exoplanets do not provide many
transits; the importance of obtaining significance at the
individual-transit level is high.
Everything we learn along these lines in working on \kt\ is very
relevant to the near-future of NASA exoplanet science with \tess.

DWH:  What are we going to learn from \ketu\ that we didn't learn from the Main Mission?  Populations!  Clusters!

DWH:  How does all this fit into the long-term goals of NASA Origins?

\section{The first Open \ketu\ Catalog}

We have made the first steps towards building an open \ketu\ catalog by
developing a method of transit search that is robust to the large systematic
effects in the \ketu\ light curves, publishing a list of planet candidates
from the first Campaign of the \ketu\ Mission, and releasing an open source
implementation of the method\footnote{\url{https://github.com/dfm/ketu}}
\citep{Foreman-Mackey:2015}.
Of the 36 published planet candidates, 18 have since been confirmed as bona
fide planets \citep{Armstrong:2015, Montet:2015}.
This transit search procedure was designed to be sensitive to the transits of
small planets even in the face of the degraded pointing of the \kepler\
spacecraft by using a data-driven, causal model for the systematic photometric
variability of the light curves.



In order to discover transiting planets, exquisite photometric precision is
crucial.
Good photometry relies on either a near-perfect flat-field
and pointing model or else data-analysis techniques that are
insensitive to these instrument properties.
The flat-field for \kepler\ was measured on the ground before the launch of
the spacecraft, but is not nearly as accurate as required to make
pointing-insensitive photometric measurements at the relevant level of
precision.
In principle direct inference of the flat-field might be possible;
however, because point sources are observed with relatively limited
spacecraft motion, and only a few percent of the data are actually stored and
downloaded to Earth, there isn't enough information in the data to derive or
infer a complete or accurate flat-field map.
Therefore, work on \KT\ is sensibly focused on building data-analysis
techniques that are pointing-insensitive.

Previous projects have developed methods to work with \KT\ data.
Both \citet{Vanderburg:2014} and \citet{Armstrong:2014}
extract aperture photometry from the pixel data
and decorrelate with image centroid position, producing light curves for each
star that are ``corrected'' for the spacecraft motion.
These data have produced the first confirmed planet found with
\KT\ \citep{Vanderburg:2015}.
Both \citet{Aigrain:2015} and \citet{Crossfield:2015} use a Gaussian Process
model for the measured flux, with pointing measurements as the inputs, and
then ``de-trend'' using the mean prediction from that model.
Other data-driven approaches have been developed and applied to the data from
space missions \citep[for example,][]{Ofir:2010, Stumpe:2012, Smith:2012,
Petigura:2013, Wang:2015} and ground-based surveys \citep[for
example,][]{Kovacs:2005, Tamuz:2005, Berta:2012} but they have yet to be
generalized to \KT.

In all of these light-curve processing methodologies, the authors follow a
traditional procedure of ``correcting'' or ``de-trending'' the light curve to
remove systematic and stellar variability as a step that happens \emph{before}
the search for transiting planets.
Fit-and-subtract is dangerous:
Small signals, such as planet transits, can be
partially absorbed into the best-fit stellar variability or systematics
models, making each individual transit event appear shallower.
In other words, the traditional methods are prone to over-fitting.
Because over-fitting will in general reduce the amplitude of true exoplanet
signals, small planets that ought to appear just above any specific
signal-to-noise or depth threshold could be missed because of the de-trending.
This becomes especially important as the amplitude of the noise increases.

The alternative to this approach is to \emph{simultaneously fit} both the
systematics and the transit signals.
Simultaneous fitting can push the detection limits to lower signal-to-noise
while robustly accounting for uncertainties about the systematic trends.
In particular, it permits us to \emph{marginalize} over choices in the noise
model and propagate any uncertainties about the systematic effects
to our confidence in the detection.
This marginalization ensures that any conclusions we come to about the
exoplanet properties are conservative, given the freedom of the systematics
model.

\paragraph{A data-driven model of photometric systematics}

To model the large systematics in the
\citet{Foreman-Mackey:2015} built a data-driven model

\section{Development and enhancements}

Hello World!  We need a better title for this section

DFM:  Laundry list here (to start)

better basis vectors

more insanely: GP of basis vectors?  Or GPLVM noise model?

prior over basis-vector amplitudes

elimination of hand vetting (progress towards that?)

better photometry of the sources; focal-plane modeling perhaps?

other shit?

\section{Population inferences}

Hello World!

\section{Prior NASA support}

Do we need this section?

\section{Culture change in exoplanet science}

This section title is too strident.

Everything in the group is done out in the open.
Refer to this proposal's github repo and tag.

\section{Management plan and data release}

The project is extremely streamlined:

The postdoctoral scholar will do most of the heavy lifting.

Foreman-Mackey will supervise on all methodological and exoplanetary
matters.

Hogg will be responsible for ensuring that catalogs get released and
papers get written.

Enhancements will proceed in parallel with Catalog generation;
Catalogs will be generated for each Campaign as it is released.
Then, each year, we will build updated catalogs for all prior
Campaigns, incorporating all enhancements to date.

Everything will be strictly versioned, so that there will be no
ambiguity about which Catalog is being used where.

We will work with NYU ITS and also \project{github} to preserve
the code and its version history in open repositories on the Web.

\clearpage
\newcommand{\arxiv}[1]{{\href{http://arxiv.org/abs/#1}{arXiv:{#1}}}}
\begin{thebibliography}{}\raggedright%

\bibitem[Aigrain \etal(2015)]{Aigrain:2015}
Aigrain, S., Hodgkin, S.~T., Irwin, M.~J., Lewis, J.~R., \& Roberts, S.~J.\
2015, \mnras, 447, 2880

\bibitem[Armstrong \etal(2014)]{Armstrong:2014}
Armstrong, D.~J., Osborn, H.~P., Brown, D.~J.~A., \etal\ 2014,
\arxiv{1411.6830}

\bibitem[Armstrong \etal(2015)]{Armstrong:2015}
Armstrong, D.~J., Veras, D., Barros, S.~C.~C., \etal\ 2015, \arxiv{1503.00692}

\bibitem[Berta \etal(2012)]{Berta:2012}
Berta, Z.~K., Irwin, J., Charbonneau, D., Burke, C.~J., \& Falco, E.~E.\ 2012,
\aj, 144, 145

\bibitem[Crossfield \etal(2015)]{Crossfield:2015}
Crossfield, I.~J.~M., Petigura, E., Schlieder, J.~E., \etal\ 2015, \apj, 804,
10

\bibitem[Foreman-Mackey \etal(2015)]{Foreman-Mackey:2015}
Foreman-Mackey, D., Montet, B.~T., Hogg, D.~W., \etal\ 2015, \arxiv{1502.04715}

\bibitem[Kov{\'a}cs \etal(2005)]{Kovacs:2005}
Kov{\'a}cs, G., Bakos, G., \& Noyes, R.~W.\ 2005, \mnras, 356, 557

\bibitem[Montet \etal(2015)]{Montet:2015}
Montet, B.~T., Morton, T.~D., Foreman-Mackey, D., \etal\ 2015,
\arxiv{1503.07866}

\bibitem[Ofir \etal(2010)]{Ofir:2010}
Ofir, A., Alonso, R., Bonomo, A.~S., \etal\ 2010, \mnras, 404, L99

\bibitem[Petigura \etal(2013)]{Petigura:2013}
Petigura, E.~A., Howard, A.~W., \& Marcy, G.~W.\ 2013,
Proceedings of the National Academy of Science, 110, 19273

\bibitem[Smith \etal(2012)]{Smith:2012}
Smith, J.~C., Stumpe, M.~C., Van Cleve, J.~E., \etal\ 2012, \pasp, 124, 1000

\bibitem[Stumpe \etal(2012)]{Stumpe:2012}
Stumpe, M.~C., Smith, J.~C., Van Cleve, J.~E., \etal\ 2012, \pasp, 124, 985

\bibitem[Tamuz \etal(2005)]{Tamuz:2005}
Tamuz, O., Mazeh, T., \& Zucker, S.\ 2005, \mnras, 356, 1466

\bibitem[Vanderburg \& Johnson(2014)]{Vanderburg:2014}
Vanderburg, A., \& Johnson, J.~A.\ 2014, \pasp, 126, 948

\bibitem[Vanderburg \etal(2015)]{Vanderburg:2015}
Vanderburg, A., Montet, B.~T., Johnson, J.~A., \etal\ 2015, \apj, 800, 59

\bibitem[Wang \etal(2015)]{Wang:2015}
Wang, D., Foreman-Mackey, D., Hogg, D.~W., \& Sch{\"o}lkopf, B.\ 2015,
American Astronomical Society Meeting Abstracts, 225, \#258.08

\end{thebibliography}

\end{document}
