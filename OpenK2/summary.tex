% This project is part of the OpenK2 project.
% Copyright 2015 the authors.

% issues
% - Catalog name? OK Catalog?  Ketu?

\documentclass[12pt]{article}
\setlength{\headheight}{0ex}
\setlength{\headsep}{0ex}
%----- exact 1-in margins
% NB: headheight and headsep MUST exist and be set
\setlength{\textwidth}{6.5in}
\setlength{\textheight}{9in}
\addtolength{\textheight}{-1.0\headheight}
\addtolength{\textheight}{-1.0\headsep}
\setlength{\topmargin}{0.0in}
\setlength{\oddsidemargin}{0.0in}
\setlength{\evensidemargin}{0.0in}

%----- typeset certain kinds of words
\newcommand{\observatory}[1]{\textsl{#1}}
\newcommand{\package}[1]{\textsf{#1}}
\newcommand{\project}[1]{\textsl{#1}}
\newcommand{\an}{\package{Astrometry.net}}
\newcommand{\NASA}{\observatory{NASA}}
\newcommand{\Kepler}{\observatory{Kepler}}
\newcommand{\kepler}{\Kepler}
\newcommand{\ketu}{\observatory{K2}}
\newcommand{\MAST}{\observatory{MAST}}
\newcommand{\EA}{\observatory{Exoplanet Archive}}
\newcommand{\TESS}{\observatory{TESS}}
\newcommand{\galex}{\observatory{GALEX}}
\newcommand{\Spitzer}{\observatory{Spitzer}}
\newcommand{\gaia}{\observatory{Gaia}}
\newcommand{\Gaia}{\gaia}
\newcommand{\lsst}{\observatory{LSST}}
\newcommand{\sdss}{\observatory{SDSS}}
\newcommand{\latin}[1]{\textit{#1}}
\newcommand{\eg}{\latin{e.g.}}
\newcommand{\etal}{\latin{et~al.}}
\newcommand{\etc}{\latin{etc.}}
\newcommand{\ie}{\latin{i.e.}}
\newcommand{\vs}{\latin{vs.}}

%----- typeset journals
% \newcommand{\aj}{Astron.\,J.}
% \newcommand{\apj}{Astrophys.\,J.}
% \newcommand{\apjl}{Astrophys.\,J.\,Lett.}
% \newcommand{\apjs}{Astrophys.\,J.\,Supp.\,Ser.}
% \newcommand{\mnras}{Mon.\,Not.\,Roy.\,Ast.\,Soc.}
% \newcommand{\aap}{Astron.\,\&~Astrophys.}

%----- Tighten up paragraphs and lists
\setlength{\parskip}{0.0ex}
\setlength{\parindent}{0.2in}
\renewenvironment{itemize}{\begin{list}{$\bullet$}{%
  \setlength{\topsep}{0.0ex}%
  \setlength{\parsep}{0.0ex}%
  \setlength{\partopsep}{0.0ex}%
  \setlength{\itemsep}{0.0ex}%
  \setlength{\leftmargin}{1.5\parindent}}}{\end{list}}
\newcounter{actr}
\renewenvironment{enumerate}{\begin{list}{\arabic{actr}.}{%
  \usecounter{actr}%
  \setlength{\topsep}{0.0ex}%
  \setlength{\parsep}{0.0ex}%
  \setlength{\partopsep}{0.0ex}%
  \setlength{\itemsep}{0.0ex}%
  \setlength{\leftmargin}{1.5\parindent}}}{\end{list}}

%----- mess with paragraph spacing!
\makeatletter
\renewcommand\paragraph{\@startsection{paragraph}{4}{\z@}%
                                    {1ex}%
                                    {-1em}%
                                    {\normalfont\normalsize\bfseries}}
\makeatother

%----- Special Hogg list for references
  \newcommand{\hogglist}{%
    \rightmargin=0in
    \leftmargin=0.25in
    \topsep=0ex
    \partopsep=0pt
    \itemsep=0ex
    \parsep=0pt
    \itemindent=-1.0\leftmargin
    \listparindent=\leftmargin
    \settowidth{\labelsep}{~}
    \usecounter{enumi}
  }

%----- side-to-side figure macro
%------- make numbers add up to 94%
 \newlength{\figurewidth}
 \newlength{\captionwidth}
 \newcommand{\ssfigure}[3]{%
   \setlength{\figurewidth}{#2\textwidth}
   \setlength{\captionwidth}{\textwidth}
   \addtolength{\captionwidth}{-\figurewidth}
   \addtolength{\captionwidth}{-0.02\figurewidth}
   \begin{figure}[htb]%
   \begin{tabular}{cc}%
     \begin{minipage}[c]{\figurewidth}%
       \resizebox{\figurewidth}{!}{\includegraphics{#1}}%
     \end{minipage} &%
     \begin{minipage}[c]{\captionwidth}%
       \textsf{\caption[]{\footnotesize {#3}}}%
     \end{minipage}%
   \end{tabular}%
   \end{figure}}

%----- top-bottom figure macro
 \newlength{\figureheight}
 \setlength{\figureheight}{0.75\textheight}
 \newcommand{\tbfigure}[2]{%
   \begin{figure}[htp]%
   \resizebox{\textwidth}{!}{\includegraphics{#1}}%
   \textsf{\caption[]{\footnotesize {#2}}}%
   \end{figure}}

%----- deal with pdf page-size stupidity
\special{papersize=8.5in,11in}
\setlength{\pdfpageheight}{\paperheight}
\setlength{\pdfpagewidth}{\paperwidth}

% specials for this proposal
\newcommand{\shortauthor}{Hogg \& Foreman-Mackey}
\newcommand{\fulltitle}{The Open \ketu\ Catalog and Exoplanet Populations}
\newcommand{\catalogname}{\observatory{The Ketu Catalog}}

% no more bad lines!
\sloppy\sloppypar

\pagestyle{empty}

\begin{document}

\noindent\textbf{\shortauthor~/~\fulltitle}
\bigskip

In a breakthrough for exoplanet research and open science, the new
\kepler\ \ketu\ Mission is delivering telemetered time-domain imaging
data on tens of thousands of stars in each of a dozen survey fields to
the public (essentially) as soon as the data have been telemetered
down by the Spacecraft.
Here we propose to build on this openness by \textbf{creating
  \thecatalog}, an exoplanet catalog built from (or updated by)
every \ketu\ Campaign data set as soon as possible after each Campaign
data release.
Importantly, we will also \textbf{release the catalog, a completeness
  analysis, all the code, and every update to any of these} to the
community immediately upon creation, as we did with a pilot catalog
made from \ketu\ Campaign~1 data.
We will also use \thecatalog\ and our expertise in hierarchical
probabilistic modeling to produce a noise-marginalized,
completeness-corrected (and geometry-corrected) inference of the true
distribution of exoplanets, given the \ketu\ and \kepler\ Main Mission
data.

In its basic form, \thecatalog\ is built by brute-force search for
periodic transit signals.
The special technology that makes it very sensitive is that the search
is done with \textbf{simultaneous fitting of spacecraft artifacts and stellar
variability}, both of which are treated as noise sources and
marginalized out.
Along with a catalog of significant exoplanet detections,
\thecatalog\ comes with a completeness analysis (false negative rate
as a function of exoplanet properties), generated by injecting
realistic artificial signals into the raw data and searching for them.

Over the course of the project, we propose to pursue a set of
\textbf{enhancements and improvements to the catalog-generation methods} and
software.
The menu of improvements includes new data-driven methods for creating
spacecraft artifact models, new models for stochastic and quasi-periodic
stellar variability, new machine-learning methods that obviate
hand-vetting of catalog results, and new physical models of the spacecraft
focal plane to improve photometry.
As we develop these enhancements, they will be propagated into annual
re-processings and re-releases of the catalog, completeness analysis,
and code.

There are several high-level goals for \thecatalog.
One is to provide targets for follow-up and discovery of interesting,
rare, or important exoplanet systems.
Our high-level goal is to understand \textbf{exoplanet populations}.
As we did with \kepler\ Main Mission data, we will use
\thecatalog\ and its associated completeness analysis to perform an
inference of the true (completeness-corrected, noise-deconvolved, and
geometry-corrected) exoplanet population.
Our technology here is \textbf{hierarchical probabilistic modeling},
with which we have been pioneers in the astrophysics community.

All software used in this project will be made public under an
open-source license and maintained in public \package{git} repositories in
contract with a reliable vendor.

\end{document}
