% This project is part of the OpenK2 project.
% Copyright 2015 the authors.

% issues
% - Catalog name? OK Catalog?  Ketu?

\documentclass[12pt]{article}
\setlength{\headheight}{0ex}
\setlength{\headsep}{0ex}
\input{hogg_nasa}
\pagestyle{empty}

\begin{document}

\noindent\textbf{\shortauthor~/~\fulltitle}
\bigskip

In a breakthrough for exoplanet research and open science, the new
\kepler\ \ketu\ Mission is delivering telemetered time-domain imaging
data on tens of thousands of stars in each of a dozen survey fields to
the public (essentially) as soon as the data have been telemetered
down by the Spacecraft (s/c).
Here we propose to build on this openness by \textbf{creating
  \thecatalog}, an exoplanet catalog built from (or updated by)
every \ketu\ Campaign data set as soon as possible after each Campaign
data release.
Importantly, we will also \textbf{release the catalog, a completeness
  analysis, all the code, and every update to any of these} to the
community immediately upon creation, as we did with a pilot catalog
made from \ketu\ Campaign~1 data.
We will also use \thecatalog\ and our expertise in hierarchical
probabilistic modeling to produce a noise-marginalized,
completeness-corrected (and geometry-corrected) inference of the true
distribution of exoplanets, given the \ketu\ and \kepler\ Main Mission
data.

In its basic form, \thecatalog\ is built by brute-force search for
periodic transit signals.
The special technology that makes it very sensitive is that the search
is done with \textbf{simultaneous fitting of s/c artifacts and stellar
variability}, both of which are treated as noise sources and
marginalized out.
Along with a catalog of significant exoplanet detections,
\thecatalog\ comes with a completeness analysis (false negative rate
as a function of exoplanet properties), generated by injecting
realistic artificial signals into the raw data and searching for them.

Over the course of the project, we propose to pursue a set of
\textbf{enhancements and improvements to the catalog-generation methods} and
software.
The menu of improvements includes new data-driven methods for creating
s/c artifact models, new models for stochastic and quasi-periodic
stellar variability, new machine-learning methods that obviate
hand-vetting of catalog results, and new physical models of the s/c
focal plane to improve photometry.
As we develop these enhancements, they will be propagated into annual
re-processings and re-releases of the catalog, completeness analysis,
and code.

There are several high-level goals for \thecatalog.
One is to provide targets for follow-up and discovery of interesting,
rare, or important exoplanet systems.
Our high-level goal is to understand \textbf{exoplanet populations}.
As we did with \kepler\ Main Mission data, we will use
\thecatalog\ and its associated completeness analysis to perform an
inference of the true (completeness-corrected, noise-deconvolved, and
geometry-corrected) exoplanet population.
Our technology here is \textbf{hierarchical probabilistic modeling},
with which we have been pioneers in the astrophysics community.

All software used in this project will be made public under an
open-source license and maintained in public \package{git} repositories in
contract with a reliable vendor.

\end{document}
